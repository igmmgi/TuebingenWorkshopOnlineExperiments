\documentclass[t]{beamer}
\setbeamercovered{invisible}

%%%%%%%%%%%%%%%%%%%%%%% Latex theme + Author and Uni Info %%%%%%%%%%%%%%%%%%%%%
\usetheme{UTedit}

\title{Online Experiments with jsPsych}
\subtitle{Introduction to jsPsych}
\author[Mackenzie]{\inst{}}
\institute[Universit\"at T\"ubingen]

%%%%%%%%%%%%%%%%%%%%%%%%%%% Presentation %%%%%%%%%%%%%%%%%%%%%%%%%%%%%%%%%%%%%%
\begin{document}
\begin{frame}
    \titlepage{}
\end{frame}

%%%%%%%%%%%%%%%%%%%%%%%%%%%%%%%%%%%%%%%%%%%%%%%%%%%%%%%%%%%%%%%%%%%%%%%%%%%%%%%
\begin{frame}[fragile]
    \frametitle{jsPsych}
    \begin{itemize}
        \item What is jsPsych?
            \begin{itemize}
                \item JavaScript library for running experiments in the browser
            \end{itemize}
        \item Useful links
            \begin{itemize}
                \item \href{https://www.jspsych.org/}{jsPsych Website}
                \item \href{https://www.github.com/jspsych/jsPsych/}{jsPsych Code}
                \item \href{https://www.pubmed.ncbi.nlm.nih.gov/24683129/}{jsPsych Paper}
                \item \href{https://www.youtube.com/watch?v=BuhfsIFRFe8}{YouTube Tutorial 1}
                \item \href{https://www.youtube.com/watch?v=8Nj3qAxY_ss}{YouTube Tutorial 2}
                \item \href{https://www.youtube.com/watch?v=LP7o0iAALik}{YouTube Tutorial 3}
            \end{itemize}
    \end{itemize}
\end{frame}


%%%%%%%%%%%%%%%%%%%%%%%%%%%%%%%%%%%%%%%%%%%%%%%%%%%%%%%%%%%%%%%%%%%%%%%%%%%%%%%
\begin{frame}[fragile]
    \frametitle{jsPsych}
    \begin{itemize}
        \item Running behavioural studies online: Is it valid?
        \item Useful references
            \begin{itemize}\scriptsize
                \item \href{https://peerj.com/articles/9414/}{Bridges, D., Pitiot, A., MacAskill, M. R., \& Peirce, J. W. (2020). The timing mega-study: comparing a range of experiment generators, both lab-based and online. PeerJ, 8, e9414.}
                \item \href{https://link.springer.com/article/10.3758/s13428-015-0567-2}{de Leeuw, Joshua R., and Benjamin A. Motz. "Psychophysics in a Web browser? Comparing response times collected with JavaScript and Psychophysics Toolbox in a visual search task." Behavior Research Methods 48.1 (2016): 1-12.}
                \item \href{https://link.springer.com/article/10.3758/s13428-015-0678-9}{Hilbig, B. E. (2016). Reaction time effects in lab-versus Web-based research: Experimental evidence. Behavior Research Methods, 48(4), 1718-1724.}
            \end{itemize}
    \end{itemize}
\end{frame}


%%%%%%%%%%%%%%%%%%%%%%%%%%%%%%%%%%%%%%%%%%%%%%%%%%%%%%%%%%%%%%%%%%%%%%%%%%%%%%%
\begin{frame}[fragile]
    \frametitle{jsPsych}
    \begin{itemize}
        \item Advantages 
            \begin{itemize}
                \item Late 2020/early 2021 only option for data collection! (Covid-19)
                \item Very quick way to collect many participants 
                \item Access different population pools (e.g., age, native language)
                    \begin{itemize}
                        \item Mechanical Turk/Prolific
                    \end{itemize}
            \end{itemize}
    \end{itemize}
\end{frame}

%%%%%%%%%%%%%%%%%%%%%%%%%%%%%%%%%%%%%%%%%%%%%%%%%%%%%%%%%%%%%%%%%%%%%%%%%%%%%%%
\begin{frame}[fragile]
    \frametitle{jsPsych}
    \begin{itemize}
        \item What do we need?
            \begin{itemize}
                \item Text Editor (Vim, VS Code, Sublime Text, R-Studio etc.)
                    \begin{itemize}
                        \item Need to edit .js (95\%), .html, and .css files
                        \item Syntax highlighting!
                    \end{itemize}
                \item Web-Browser  
                    \begin{itemize}
                        \item Need to test on most commonly used browsers (e.g., Firefox, Chrome, and Safari)
                    \end{itemize}
                \item jsPsych library
                \item Web Server (e.g., Pavlovia)
                    \begin{itemize}
                        \item Not required for local development/initial testing
                    \end{itemize}
                \item Git (required for interaction with Pavlovia + useful in general for code development) 
                    \begin{itemize}
                        \item \href{https://git-scm.com/}{Git link}
                    \end{itemize}
            \end{itemize}
    \end{itemize}
\end{frame}


%%%%%%%%%%%%%%%%%%%%%%%%%%%%%%%%%%%%%%%%%%%%%%%%%%%%%%%%%%%%%%%%%%%%%%%%%%%%%%%
\begin{frame}[fragile]
    \frametitle{Git}
    \begin{itemize}
        \item What is Git?
            \begin{itemize}
                \item Git is version control software
                    \begin{itemize}
                        \item We can use it to keep track of changes in our experiment code (complete history of changes)
                        \item Avoid need for myexperiment180121.js, myexperiment190121\_test\_change.js, myexperiment190121\_other\_change.js, and so on
                        \item Makes collaboration easier (share code, use code from others)
                    \end{itemize}
            \end{itemize}
        \item What is GitHub/GitLab
            \begin{itemize}
                \item Two separate online hosts for Git projects
                    \begin{itemize}
                        \item \href{https://github.com/}{GitHub}
                        \item \href{https://gitlab.com/}{GitLab}
                    \end{itemize}
            \end{itemize}
    \end{itemize}
\end{frame}


%%%%%%%%%%%%%%%%%%%%%%%%%%%%%%%%%%%%%%%%%%%%%%%%%%%%%%%%%%%%%%%%%%%%%%%%%%%%%%%
\begin{frame}[fragile]
    \frametitle{Git Basics: Walk-through I}
    \begin{itemize}
        \item Create a new project (local computer)
            \begin{itemize}
                \item README.md file
                \item git init . directory
                \item git add .
                \item git status 
                \item git commit 
            \end{itemize}
        \item Create a repository on GitHub\footnotemark or GitLab
            \begin{itemize}
                \item Your account $\rightarrow$ Your repositories $\rightarrow$ New 
                \item Repository name $\rightarrow$ Create repository
                \item Option $\rightarrow$ …or push an existing repository from the command line
            \end{itemize}
    \end{itemize}
    \footnotetext[1]{\tiny Instructions refer to GitHub}
\end{frame}


%%%%%%%%%%%%%%%%%%%%%%%%%%%%%%%%%%%%%%%%%%%%%%%%%%%%%%%%%%%%%%%%%%%%%%%%%%%%%%%
\begin{frame}[fragile]
    \frametitle{Git Basics: Walk-through II}
    \begin{itemize}
        \item Upload our local repository to GitHub or GitLab
            \begin{itemize}
                \item git remote add origin https://github.com/igmmgi/XXX.git
                \item git branch -M main\footnotemark
                \item git push -u origin main
            \end{itemize}
        \item Locate project to clone (on GitHub/GitLab)
            \begin{itemize}
                \item Code $\rightarrow$ Copy/Paste
            \end{itemize}
        \item Clone an existing project (local computer)
            \begin{itemize}
                \item git clone XXX 
                \item git log
            \end{itemize}
        \item Clone TuebingenWorkshopOnlineExperiments which contains the course materials
            \begin{itemize}\scriptsize
                \item git clone https://github.com/igmmgi/TuebingenWorkshopOnlineExperiments.git
                \item git pull
            \end{itemize}
    \end{itemize}
    \footnotetext[2]{\tiny master to main name change 2020/2021}
\end{frame}


%%%%%%%%%%%%%%%%%%%%%%%%%%%%%%%%%%%%%%%%%%%%%%%%%%%%%%%%%%%%%%%%%%%%%%%%%%%%%%%
\begin{frame}[fragile]
    \frametitle{jsPsych: Getting Started}
    \begin{itemize}
        \item Three related technologies 
            \begin{itemize}
                \item HTML (Hypertext Markup Language) with file extension .html 
                    \begin{itemize}
                        \item Controls the content on the webpage
                    \end{itemize}
                \item CSS (Cascading Style Sheets) with file extendion .css
                    \begin{itemize}
                        \item Controls the style on the webpage
                    \end{itemize}
                \item JavaScript with file extension .js
                    \begin{itemize}
                        \item Used to add some interaction
                    \end{itemize}
            \end{itemize}
    \end{itemize}
\end{frame}


%%%%%%%%%%%%%%%%%%%%%%%%%%%%%%%%%%%%%%%%%%%%%%%%%%%%%%%%%%%%%%%%%%%%%%%%%%%%%%%
\begin{frame}[fragile]
    \frametitle{HTML + CSS + javascript}
    \begin{itemize}
        \item Useful resources
            \begin{itemize}
                \item \href{https://www.w3schools.com/html/default.asp}{w3schools.com (HTML)}
                \item \href{https://www.w3schools.com/css/default.asp}{w3schools.com (CSS)}
                \item \href{https://www.w3schools.com/js/DEFAULT.asp}{w3schools.com (javascript)}
            \end{itemize}
        \item Demo Files
            \begin{itemize}
                \item example.html
                \item example\_with\_inline\_css.html
                \item example\_with\_spearate\_css\_file.html and example.css
                \item example\_with\_javascript.html
            \end{itemize}
    \end{itemize}
\end{frame}


%%%%%%%%%%%%%%%%%%%%%%%%%%%%%%%%%%%%%%%%%%%%%%%%%%%%%%%%%%%%%%%%%%%%%%%%%%%%%%%
\begin{frame}[fragile]
    \frametitle{jsPsych basics}
    \begin{itemize}
        \item Combination of javascript, html, css
        \item Specific high-level code for behavioural experiments 
            \begin{itemize}
                \item Present text/images/sounds/movies
                \item Record key-presses, reaction times, slider responses etc.
                \item Organise data 
                \item Randomisation procedures
            \end{itemize}
        \item Built around the idea pre-defined trial-types or plugins
            \begin{itemize}
                \item Easy to use 
                \item Requires very little actual coding
                \item Covers a wide-rage of use cases 
                \item We can also create custom plugins for more specific experiments (requires a little bit of coding)
            \end{itemize}
    \end{itemize}
\end{frame}


%%%%%%%%%%%%%%%%%%%%%%%%%%%%%%%%%%%%%%%%%%%%%%%%%%%%%%%%%%%%%%%%%%%%%%%%%%%%%%%
\begin{frame}[fragile]
    \frametitle{jsPsych: A first ``experiment"}
    \begin{itemize}
        \item Demo Files
            \begin{itemize}
                \item jspsych\_exp1.html \& jspsych\_exp1.js
                \item jspsych\_exp2.html \& jspsych\_exp2.js
                \item jspsych-6.2.0/examples/
            \end{itemize}
    \end{itemize}
\end{frame}


%%%%%%%%%%%%%%%%%%%%%%%%%%%%%%%%%%%%%%%%%%%%%%%%%%%%%%%%%%%%%%%%%%%%%%%%%%%%%%%
\begin{frame}[fragile]
    \frametitle{jsPsych: Posner Task}
    \begin{itemize}
        \item Files
            \begin{itemize}
                \item TuebingenWorkshopOnlineExperiments/jsPsych/posner\_task
            \end{itemize}
        \item Walk-through ... 
    \end{itemize}
\end{frame}

%%%%%%%%%%%%%%%%%%%%%%%%%%%%%%%%%%%%%%%%%%%%%%%%%%%%%%%%%%%%%%%%%%%%%%%%%%%%%%%
\begin{frame}[fragile]
    \frametitle{jsPsych: Posner Task (Pavlovia)}
    \begin{itemize}
        \item Useful link
            \begin{itemize}
                \item \href{https://pavlovia.org/js-psych}{Pavlovia Instructions for jsPsych}
            \end{itemize}
        \item Walk-through ... 
    \end{itemize}
\end{frame}

%%%%%%%%%%%%%%%%%%%%%%%%%%%%%%%%%%%%%%%%%%%%%%%%%%%%%%%%%%%%%%%%%%%%%%%%%%%%%%%
\end{document}
